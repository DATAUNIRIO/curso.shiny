\documentclass[12pt, compress, usetitleprogressbar]{beamer}\usepackage[]{graphicx}\usepackage[]{color}
%% maxwidth is the original width if it is less than linewidth
%% otherwise use linewidth (to make sure the graphics do not exceed the margin)
\makeatletter
\def\maxwidth{ %
  \ifdim\Gin@nat@width>\linewidth
    \linewidth
  \else
    \Gin@nat@width
  \fi
}
\makeatother

\definecolor{fgcolor}{rgb}{0, 0, 0}
\newcommand{\hlnum}[1]{\textcolor[rgb]{0,0,0}{#1}}%
\newcommand{\hlstr}[1]{\textcolor[rgb]{0,0,0}{#1}}%
\newcommand{\hlcom}[1]{\textcolor[rgb]{0.4,0.4,0.4}{\textit{#1}}}%
\newcommand{\hlopt}[1]{\textcolor[rgb]{0,0,0}{\textbf{#1}}}%
\newcommand{\hlstd}[1]{\textcolor[rgb]{0,0,0}{#1}}%
\newcommand{\hlkwa}[1]{\textcolor[rgb]{0,0,0}{\textbf{#1}}}%
\newcommand{\hlkwb}[1]{\textcolor[rgb]{0,0,0}{\textbf{#1}}}%
\newcommand{\hlkwc}[1]{\textcolor[rgb]{0,0,0}{\textbf{#1}}}%
\newcommand{\hlkwd}[1]{\textcolor[rgb]{0,0,0}{\textbf{#1}}}%
\let\hlipl\hlkwb

\usepackage{framed}
\makeatletter
\newenvironment{kframe}{%
 \def\at@end@of@kframe{}%
 \ifinner\ifhmode%
  \def\at@end@of@kframe{\end{minipage}}%
  \begin{minipage}{\columnwidth}%
 \fi\fi%
 \def\FrameCommand##1{\hskip\@totalleftmargin \hskip-\fboxsep
 \colorbox{shadecolor}{##1}\hskip-\fboxsep
     % There is no \\@totalrightmargin, so:
     \hskip-\linewidth \hskip-\@totalleftmargin \hskip\columnwidth}%
 \MakeFramed {\advance\hsize-\width
   \@totalleftmargin\z@ \linewidth\hsize
   \@setminipage}}%
 {\par\unskip\endMakeFramed%
 \at@end@of@kframe}
\makeatother

\definecolor{shadecolor}{rgb}{.97, .97, .97}
\definecolor{messagecolor}{rgb}{0, 0, 0}
\definecolor{warningcolor}{rgb}{1, 0, 1}
\definecolor{errorcolor}{rgb}{1, 0, 0}
\newenvironment{knitrout}{}{} % an empty environment to be redefined in TeX

\usepackage{alltt}

\usetheme{metropolis}

\usepackage{algorithm}
\usepackage{algpseudocode}
\usepackage{booktabs}
\usepackage[scale=2]{ccicons}
\usepackage{minted}
\usepackage{amssymb}
\usepackage{amsfonts}
\usepackage{amsmath}
%\usepackage{amssymb}
\usepackage{amsthm}
\usepackage[portuges]{babel}
%\usepackage[portuges]{polyglossia}
%\usepackage{polyglossia}
\usepackage{bm}
\usepackage{enumitem}
\usepackage{icomma}
\usepackage[retainorgcmds]{IEEEtrantools}
\usepackage[showonlyrefs]{mathtools}
\usepackage{media9}
\usepackage{multirow}
\usepackage{natbib}
\usepackage{shadow}
\usepackage{subfigure}
\usepackage{tikz}
\usetikzlibrary{shapes,arrows}
\usepackage{verbatim}
\usepackage{xcolor}

%\usepgfplotslibrary{dateplot}

\usemintedstyle{trac}


%%% Identification

\title{Introdução ao Shiny}
\subtitle{Como Criar Aplicativos Web Utilizando o R}
\date{23 e 24 de Outubro de 2018}
\author{Marcus Nunes}
\institute{Universidade Federal do Rio Grande do Norte}



%%% New colors

\definecolor{mDarkBrown}{HTML}{604c38}
\definecolor{mDarkTeal}{HTML}{23373b}

\definecolor{mLightBrown}{HTML}{EB811B}
\definecolor{mMediumBrown}{HTML}{C87A2F}

\definecolor{mLightRed}{HTML}{EB331B}
\definecolor{mLightGreen}{HTML}{14B03D}
\definecolor{mLightTeal}{HTML}{137D91}



% Define block styles

\tikzstyle{block} = [rectangle, draw, fill=mDarkTeal, 
    text width=6em, text centered, rounded corners, minimum height=4em]
\tikzstyle{line} = [draw, -latex']
\tikzstyle{blockFeatured} = [rectangle, draw, fill=mLightTeal, 
    text width=6em, text centered, rounded corners, minimum height=4em]


% Translations

\deftranslation[to=portuges]{Theorem}{Teorema}
\deftranslation[to=portuges]{theorem}{teorema}
\deftranslation[to=portuges]{Definition}{Definição}
\deftranslation[to=portuges]{definition}{definição}
\deftranslation[to=portuges]{Corollary}{Corolário}
\deftranslation[to=portuges]{corollary}{corolário}
\deftranslation[to=portuges]{Proof}{Prova}
\deftranslation[to=portuges]{proof}{prova}
\deftranslation[to=portuges]{Example}{Exemplo}
\deftranslation[to=portuges]{example}{exemplo}


%\setmainlanguage{portuguese}
\IfFileExists{upquote.sty}{\usepackage{upquote}}{}
\begin{document}


% new commands

\newcommand{\alphahat}{\widehat{\alpha}}
\newcommand{\amostra}{x_1, x_2, \cdots, x_n}
\newcommand{\Amostra}{X_1, X_2, \cdots, X_n}
\newcommand{\AmostraY}{Y_1, Y_2, \cdots, Y_m}
\newcommand{\AmostraYn}{Y_1, Y_2, \cdots, Y_n}
\newcommand{\Bin}{\mbox{Bin}}
\newcommand{\Dbar}{\overline{D}}
\newcommand{\dbar}{\overline{d}}
\newcommand{\E}{E}
\newcommand{\EP}{\mbox{EP}}
\newcommand{\EPe}{\widehat{\mbox{EP}}}
\newcommand{\EQM}{\mbox{EQM}}
\newcommand{\EMV}{\mbox{EMV}}
\newcommand{\ErroPadrao}{\mbox{Erro Padrão}}
\newcommand{\Estimativa}{\mbox{Estimativa}}
\newcommand{\Exp}{\mbox{Exp}}
\newcommand{\IC}{\mbox{IC}}
\newcommand{\integral}{\int_{-\infty}^{\infty}}
\newcommand{\LL}{\mathcal{L}}
\newcommand{\limite}{\lim_{n \rightarrow \infty}}
\newcommand{\Med}{\mbox{Med}}
\newcommand{\Multiplicador}{\mbox{Multiplicador}}
\newcommand{\p}{\widehat{p}}
\newcommand{\parametros}{\theta_1, \cdots, \theta_r}
\newcommand{\parametroshat}{\widehat{\theta_1}, \cdots, \widehat{\theta_r}}
\newcommand{\R}{\mathbb{R}}
\newcommand{\sigmahat}{\widehat{\sigma}}
\newcommand{\specialcell}[2][t]{%
  \begin{tabular}[#1]{@{}c@{}}#2\end{tabular}}
\newcommand{\soma}{X_1 + X_2 + \cdots + X_n}
\newcommand{\soman}{\sum_{i=1}^n}
\newcommand{\talpha}{t_{(gl; \alpha/2)}}
\newcommand{\thetahat}{\widehat{\theta}}
\newcommand{\Unif}{\mbox{Unif}}
\newcommand{\Var}{\mbox{Var}}
\newcommand{\Weibull}{\mbox{Weibull}}
\newcommand{\Xb}{\bm{X}}
\newcommand{\xb}{\bm{x}}
\newcommand{\Xbar}{\overline{X}}
\newcommand{\XbarCum}{\overline{X}_{c_1}}
\newcommand{\XbarCdois}{\overline{X}_{c_2}}
\newcommand{\xbar}{\overline{x}}
\newcommand{\Yb}{\bm{Y}}
\newcommand{\Ybar}{\overline{Y}}
\newcommand{\Yhat}{\widehat{Y}}
\newcommand{\zalpha}{z_{\alpha/2}}
\newcommand{\Zbarra}{\overline{Z}}
\newcommand{\zg}{z_{\gamma}}

%%%%%%%%%%%%%%%%%%%%%%




% knitr options







%%%%%%%%%%%%%%%%%%%%%%


\maketitle





%%%%%%%%%%%%%
%%% SLIDE %%%

\plain{Introdução}

%%%%%%%%%%%%%





%%%%%%%%%%%%%
%%% SLIDE %%%

\begin{frame}

\frametitle{Introdução}

\begin{enumerate}[label=$\bullet$, leftmargin=*]

  \item \texttt{shiny} é um pacote do R com um framework para criação de aplicativos web

  \item Ele permite que pessoas com pouca experiência em programação web consigam criar sites dinâmicos utilizando seus conhecimentos em R

  \item Algumas aplicações feitas com o \texttt{shiny} podem ser vistas em \url{http://shiny.estatistica.ccet.ufrn.br}

  \item O conteúdo deste curso está disponível no endereço \url{https://github.com/mnunes/curso.shiny/}

\end{enumerate}

\end{frame}

%%%%%%%%%%%%%





%%%%%%%%%%%%%
%%% SLIDE %%%

\plain{Instalação do Shiny}

%%%%%%%%%%%%%







%%%%%%%%%%%%%
%%% SLIDE %%%

\begin{frame}[fragile]

\frametitle{Anatomia de Um \emph{Shiny App}}

\begin{enumerate}[label=$\bullet$, leftmargin=*]

  \item Como todo pacote do \texttt{R}, o \texttt{shiny} pode ser instalado a partir do prompt através do comando

\begin{knitrout}\footnotesize
\definecolor{shadecolor}{rgb}{0.98, 0.98, 0.98}\color{fgcolor}\begin{kframe}
\begin{alltt}
\hlstd{> }\hlkwd{install.packages}\hlstd{(}\hlstr{"shiny"}\hlstd{)}
\end{alltt}
\end{kframe}
\end{knitrout}

  \item Ao rodar

\begin{knitrout}\footnotesize
\definecolor{shadecolor}{rgb}{0.98, 0.98, 0.98}\color{fgcolor}\begin{kframe}
\begin{alltt}
\hlstd{> }\hlkwd{library}\hlstd{(shiny)}
\end{alltt}
\end{kframe}
\end{knitrout}

  \noindent o pacote estará carregado e pronto para uso

  %\item Não é necessário se preocupar com 

\end{enumerate}

\end{frame}

%%%%%%%%%%%%%








%%%%%%%%%%%%%
%%% SLIDE %%%

\plain{Anatomia de Um Shiny App}

%%%%%%%%%%%%%








%%%%%%%%%%%%%
%%% SLIDE %%%

\begin{frame}

\frametitle{Anatomia de Um \emph{Shiny App}}

\begin{enumerate}[label=$\bullet$, leftmargin=*]

  \item Todo \emph{shiny app} é composto de até três partes:

  \begin{enumerate}[label=$\bullet$, leftmargin=*]

    \item \texttt{ui.R}: é onde a interface com o usuário (user interface) é definida

    \item \texttt{server.R}: os comandos do R que são a alma do app estão neste arquivo, ou seja, é aqui que os gráficos são construídos, que dados são filtrados etc.

    \item \texttt{global.R}: serve para organizar o carregamento de pacotes, conjuntos de dados e tudo o que necessitar ser acessado de maneira global pelo app

  \end{enumerate}

  \item Enquanto os arquivos \texttt{server.R} e \texttt{ui.R} são obrigatórios, o arquivo \texttt{global.R} é opcional

\end{enumerate}

\end{frame}

%%%%%%%%%%%%%







\begin{comment}



%%%%%%%%%%%%%
%%% SLIDE %%%

\begin{frame}

\frametitle{Anatomia de Um Shiny App}

\begin{enumerate}[label=$\bullet$, leftmargin=*]

  \item Vamos ver como isso funciona na prática

  \item Abra todos os três arquivos da pasta \texttt{exemplos/01-kmeans/}

\end{enumerate}

\end{frame}

%%%%%%%%%%%%%







%%%%%%%%%%%%%
%%% SLIDE %%%

\begin{frame}

\frametitle{Anatomia de Um Shiny App}

\begin{enumerate}[label=$\bullet$, leftmargin=*]

  \item Vamos fazer algumas alterações em um aplicativo mais simples

  \item Abra todos os três arquivos da pasta \texttt{exemplos/02-histograma/}

\end{enumerate}

\end{frame}

%%%%%%%%%%%%%

\end{comment}




%%%%%%%%%%%%%
%%% SLIDE %%%

\begin{frame}

\frametitle{Anatomia de Um Shiny App}

\begin{enumerate}[label=$\bullet$, leftmargin=*]

  \item Vamos ver como isso funciona na prática

  \item Abra os arquivos \texttt{ui.R}, \texttt{server.R} e \texttt{global.R} presentes na pasta \texttt{exemplos/01-histograma/}

  \item A melhor maneira de aprender como utilizar o \texttt{shiny} é fazendo algumas alterações em um aplicativo mais simples e vendo como estas alterações se comportam

\end{enumerate}

\end{frame}

%%%%%%%%%%%%%











%%%%%%%%%%%%%
%%% SLIDE %%%

\begin{frame}

\frametitle{Anatomia de Um \emph{Shiny App}}

\begin{enumerate}[label=$\bullet$, leftmargin=*]

  \item Programas simples podem ser rodados em apenas um arquivo

  \item Podemos colocar os códigos presentes em \texttt{ui.R}, \texttt{server.R} e \texttt{global.R} em somente um local

  \item Abra o arquivo \texttt{exemplos/01-histograma/hist-simples.R} para ver como isto é feito

\end{enumerate}

\end{frame}

%%%%%%%%%%%%%






\begin{comment}

%%%%%%%%%%%%%
%%% SLIDE %%%

\begin{frame}[fragile]

\frametitle{Anatomia de Um \emph{Shiny App}}



\end{frame}

%%%%%%%%%%%%%

\end{comment}







%%%%%%%%%%%%%
%%% SLIDE %%%

\plain{Tipos de Layout}

%%%%%%%%%%%%%





%%%%%%%%%%%%%
%%% SLIDE %%%

\begin{frame}

\frametitle{Tipos de Layout}

\begin{enumerate}[label=$\bullet$, leftmargin=*]

  \item O \texttt{shiny} já vem com diversos layouts pré-configurados para que possamos criar nossas ferramentas

  \item Basta escolher um deles e começar a produzir o nosso app

\end{enumerate}

\end{frame}

%%%%%%%%%%%%%





%%%%%%%%%%%%%
%%% SLIDE %%%

\begin{frame}

\frametitle{Tipos de Layout}

\begin{enumerate}[label=$\bullet$, leftmargin=*]

  \item \texttt{sidebarLayout}

  \item \texttt{splitLayout}

  \item \texttt{verticalLayout}

  \item \texttt{flowLayout}

\end{enumerate}

\end{frame}

%%%%%%%%%%%%%










%%%%%%%%%%%%%
%%% SLIDE %%%

\plain{Tipos de Interação}

%%%%%%%%%%%%%






%%%%%%%%%%%%%
%%% SLIDE %%%

\begin{frame}

\frametitle{Tipos de Interação}

\begin{enumerate}[label=$\bullet$, leftmargin=*]

  \item Já vimos algumas maneiras de interagir com os apps utilizando

  \item Estas maneiras não são as únicas de criarmos formas de interação com nossos programas

  \item Abra os arquivos da pasta \texttt{03-inputs} para que exploremos elas

\end{enumerate}

\end{frame}

%%%%%%%%%%%%%





%%%%%%%%%%%%%
%%% SLIDE %%%

\begin{frame}

\frametitle{Tipos de Interação}

\begin{enumerate}[label=$\bullet$, leftmargin=*]

  \item \texttt{checkboxInput}: cria uma caixa de seleção com apenas uma opção

  \item \texttt{checkboxGroupInput}: cria uma caixa de seleção com mais de uma opção

  \item \texttt{dateInput}: abre um calendário para a seleção de datas

\end{enumerate}

\end{frame}

%%%%%%%%%%%%%





%%%%%%%%%%%%%
%%% SLIDE %%%

\begin{frame}

\frametitle{Tipos de Interação}

\begin{enumerate}[label=$\bullet$, leftmargin=*]

  \item \texttt{textInput}: cria uma caixa de texto

  \item \texttt{numericInput}: cria uma caixa que recebe apenas números

  \item \texttt{passwordInput}: cria uma caixa de texto para receber senhas
  
\end{enumerate}

\end{frame}

%%%%%%%%%%%%%





%%%%%%%%%%%%%
%%% SLIDE %%%

\begin{frame}

\frametitle{Tipos de Interação}

\begin{enumerate}[label=$\bullet$, leftmargin=*]

  \item \texttt{selectInput}: cria uma caixa com uma lista de seleção

  \item \texttt{sliderInput}: cria uma maneira de selecionar um valor numérico em um intervalo
  
\end{enumerate}

\end{frame}

%%%%%%%%%%%%%





%%%%%%%%%%%%%
%%% SLIDE %%%

\begin{frame}

\frametitle{Tipos de Interação}

\begin{enumerate}[label=$\bullet$, leftmargin=*]

  \item 
  
\end{enumerate}

\end{frame}

%%%%%%%%%%%%%









%%%%%%%%%%%%%
%%% SLIDE %%%

\plain{Nosso Primeiro Projeto}

%%%%%%%%%%%%%








%%%%%%%%%%%%%
%%% SLIDE %%%

\begin{frame}

\frametitle{Nosso Primeiro Projeto}

\begin{enumerate}[label=$\bullet$, leftmargin=*]

  \item Não há como iniciar um projeto sem sabermos onde queremos chegar

  \item Por isso, é importante definirmos qual o objetivo 

  \item Eu tenho uma proposta: criar um dashboard para análise de pokémons

\end{enumerate}

\end{frame}

%%%%%%%%%%%%%








%%%%%%%%%%%%%
%%% SLIDE %%%

\begin{frame}

\frametitle{Nosso Primeiro Projeto}

\begin{enumerate}[label=$\bullet$, leftmargin=*]

  \item \alert{Dashboard} é uma página que exibe informações importantes sobre algum assunto de interesse

  \item Estas informações vão de coisas simples, como um gráfico de linha com o total de vendas anuais de uma empresa, até informações complexas, como o mapa de calor das vendas de acordo com a sua localização geográfica

  \item Em um dashboard são exibidos tabelas, gráficos e mecanismos de controle e personalização das informações

  \item Assim, em vez de criarmos milhares de relatórios personalizados, deixamos que o usuário decida que informações ele deseja 

\end{enumerate}

\end{frame}

%%%%%%%%%%%%%













%%%%%%%%%%%%%
%%% SLIDE %%%

\plain{Melhorando a Apresentação}

%%%%%%%%%%%%%










%%%%%%%%%%%%%
%%% SLIDE %%%

\begin{frame}

\frametitle{Melhorando a Apresentação}

\begin{enumerate}[label=$\bullet$, leftmargin=*]

  \item 

\end{enumerate}

\end{frame}

%%%%%%%%%%%%%











%%%%%%%%%%%%%
%%% SLIDE %%%

\plain{Considerações Finais}

%%%%%%%%%%%%%



%%%%%%%%%%%%%
%%% SLIDE %%%

\begin{frame}

\frametitle{Considerações Finais}

\begin{enumerate}[label=$\bullet$, leftmargin=*]

  \item 

\end{enumerate}

\end{frame}

%%%%%%%%%%%%%










\end{document}
